\label{sec:contributors-reference}
\subsection{Writing code}
This section contains useful information for contributors interested in modifying the code base.

\subsubsection{Setting up the application}
To contribute to \texttt{Authored by AI}, you have to setup a working development environment. The recommended method to check if your installation of the project works as expected is to run it and use features looking for eventual failures related to your setup. To quickly set up \texttt{ABAI} check out the \hyperref[subsec:setup]{setup} guide.

\subsubsection{Writing a patch}
To write a \texttt{patch}, make the change that you want in the code base. Please be careful to respect the \hyperref[subsubsec:coding-style]{coding style}. There is a \hyperref[subsec:topic-guide]{topic guide} that explains what conventions are in use.\\

\begin{tcolorbox}[colback=blue!10!white,colframe=blue!70!black,title=Note]
If you have troubles finding in which file to make changes, try reading the topic guide about the \hyperref[subsubsec:project-structure]{project structure}
\end{tcolorbox}

\subsubsection{Test the patch}
For a modification to be considered and added to future releases, it has to be tested and approved with acceptable stability. It will then be added to the next release. To run the unit tests of the application, please refer to the \hyperref[subsec:testing]{\texttt{testing}} section.
