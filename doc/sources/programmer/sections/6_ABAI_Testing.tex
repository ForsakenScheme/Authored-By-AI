\subsection{How to run unit tests?}
Tests are performed using the \href{https://docs.python.org/3/library/unittest.html}{unittest} module of the standard \texttt{Python} library.

\subsection{Requirements}
To run the unit tests, you need to have a working development environment. Please refer to the \hyperref[sec:setup-guide]{setup guide} if this is not yet the case. 

\subsection{How to write a unit test?}
If you want to write a unit test to test out a feature, please proceed like this :
\begin{enumerate}
    \item If a test module doesn't yet exist for the feature you need to test, create a new test module named \texttt{\textbf{test\_[module name]}} with the module name being the name of the module that contains the feature to test. \\
    
    \item Next, write your test method named like this : \textbf{\texttt{test\_[feature name]}}. The feature name is the name of the method you need to test.
\end{enumerate}
\subsection{Unit testing}
To run the unit tests, run the following command at the root of the project: 
\begin{codebox}
    \large\texttt{\$ python -m unittest discover -s code/backend/tests -p "test\_*.py" -v}
\end{codebox}
\clearpage
\subsection{Test coverage}
Tests coverage analysis is performed using the \href{https://coverage.readthedocs.io/en/7.5.1/}{coverage} module.
To perform the coverage analysis use the following commands sequentially at the root of the project:
\begin{codebox}
    \large\texttt{\$ coverage run -m unittest discover -s code/backend/tests -p "test\_*.py"}\\
    \large\texttt{\$ coverage html}
\end{codebox}
\vspace{0.25cm}
After this, you should be able to visualize the coverage analysis result by opening the generated \textbf{\texttt{index.html}} file from the \textbf{\texttt{htmlcov}} folder. 