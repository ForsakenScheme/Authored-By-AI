\subsection{Contribution to ABAI}
If you want to contribute to \texttt{Authored by AI}, this is the place to start.

\subsubsection{Required skills}

Authored by AI is majorily written in \href{https://www.python.org/}{\texttt{Python}}. Python knowledge would therefore be a precious asset to you and will make it easier for you to contribute to the source code. Furthermore, \href{https://www.djangoproject.com/}{\texttt{Django}} is used as a framework for the front-end development of the web interface.

In addition, it could be beneficial to have affinities with the following packages used throughout the project :
\begin{itemize}
    \item \href{https://scikit-learn.org/stable/}{\texttt{Scikit-learn}}
    \item \href{https://www.nltk.org/index.html}{\texttt{NLTK}}
    \item \href{https://numpy.org/}{Numpy}
    \item \href{https://www.riverbankcomputing.com/software/pyqt/intro}{PyQt}
\end{itemize}

The \texttt{User guide} and \texttt{Programmer guide} are written in \href{https://www.latex-project.org/}{\texttt{LaTeX}}, therefore, if you want to contribute to the documentation, it would be good to have some familiarity with it.

\subsubsection{Where to start?}
If you are still interested in contributing to Authored by AI, you will find guidance on how to do so by reading the following sections of the guide: 
\begin{itemize}
    \item \hyperref[sec:contributors-reference]{Contributor's reference}: Contains useful references for any contributor of ABAI. 
    \item \hyperref[sec:setup-guide]{Setup guide}: Explains how to set up your development environment.
    \item \hyperref[sec:source-code-documentation]{Source code documentation}: Understanding the module distribution. 
    \item \hyperref[subsec:project-structure]{Project structure}: Explains how the project is structured.
\end{itemize}


